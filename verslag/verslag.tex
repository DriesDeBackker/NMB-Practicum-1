%%%%%%%%%%%%%%%%%%%%%%%%%%%%%%%%%%%%%%%%%%%%%%%%%%%%%%%%%%%%%%%%%%%%%%%%%%%
%%%                                                                     %%%
%%%   LaTeX template voor het verslag van P&O: Computerwetenschappen.   %%%
%%%                                                                     %%%
%%%   Opties:                                                           %%%
%%%     tt      Tussentijdsverslag                                      %%%
%%%     eind    Eindverslag                                             %%%
%%%                                                                     %%%
%%%   4 november 2013                                                   %%%
%%%   Versie 1.1                                                        %%%
%%%                                                                     %%%
%%%%%%%%%%%%%%%%%%%%%%%%%%%%%%%%%%%%%%%%%%%%%%%%%%%%%%%%%%%%%%%%%%%%%%%%%%%

\documentclass[een]{practicumverslag}
\def\thesubsubsection{\alph{subsubsection}}
\setlength\parindent{0pt}

%%% PACKAGES %%%
\usepackage{lipsum}
\usepackage{subcaption}
\usepackage{amsfonts}
\usepackage{mathtools}


\begin{document}

% == TITELPAGINA == %
\members{Stijn Kuypers\\
         Dries De Backker} % teamleden
         
\maketitlepage


\pagebreak

% == Hier volgt de uitwerking van deel 1 == %
\section{Projectoren en de QR factorisatie}

\subsection*{Opgave 1}

Zij $P \in \mathbb{R}^{n \times n} $ en $x \in \mathbb{R}^{n}$, dan $Px = \frac{x + Fx}{2}$ met $F \in \mathbb{R}^{n \times n}$ de matrix die de elementen $x_1$ en $x_2$, $x_3$ en $x_4$, ... omwisselt bij de vermenigvuldiging $Fx.$
Dan:

\begin{equation}
	Px = \frac{1}{2}(I + F)x \Leftrightarrow P = \frac{1}{2}(I+F)
\end{equation}
	
* $P$ is een projector indien $P$ vierkant(gegeven) en $P$ idempotent ($P^2 = P$).
We hebben:

\begin{equation}
	P^2 = \frac{1}{4}(I^2 + 2F + F^2)
\end{equation}

Aangezien $F$ het eerste en tweede element, het derde en vierde element enz... van een vector omwisselt, doet een tweede vermenigvuldiging met $F$ deze permutatie terug teniet. 
Dus: $F^2 = I$ . Dan:

\begin{equation}
	P^2 = \frac{1}{4}(I + 2F + I) = \frac{1}{4}(2I + 2F) = \frac{1}{2}(I + F) = P
\end{equation}

* $P$ is een orthogonale projector $\Leftrightarrow P = P^{*}$.
We hebben:

\begin{equation}
	P^* = \frac{1}{2}(I + F)^* = \frac{1}{2}(I^* + F^*)
\end{equation}

We kunnen nagaan dat:

\begin{equation}
	F = \begin{bmatrix}
		0 & 1 &  \\
  		1 & 0 &  \\
  		  &   & 0 & 1\\
  		  &   & 1 & 0\\
  		  &   &   &  & \ddots &  \\
  		  &   &   &  & 	 & 0 & 1 \\
  		  &   &   &  &   & 1 & 0 \\
		\end{bmatrix} 
\end{equation}
zodat:
\begin{equation}
		F^* = \begin{bmatrix}
		0 & 1 &  \\
  		1 & 0 &  \\
  		  &   & 0 & 1\\
  		  &   & 1 & 0\\
  		  &   &   &  & \ddots &  \\
  		  &   &   &  & 	 & 0 & 1 \\
  		  &   &   &  &   & 1 & 0 \\
		\end{bmatrix} 
\end{equation}
en dus: 
\begin{equation}
	P^* = \frac{1}{2}(I^* + F^*) = \frac{1}{2}(I + F) = P
\end{equation}
waaruit volgt dat P een orthogonale projector is.

\pagebreak


\subsection*{Opgave 2}

Een Householder transformatiematrix heeft de volgende structuur:

\begin{equation}
		Q_k = \begin{bmatrix}
		I &  \\
  		  & F   \\
		\end{bmatrix} 
\end{equation}
met $I$ de $(k-1) \times (k-1)$ eenheidsmatrix en $F$ een $(m-k+1) \times (m-k+1)$ unitaire Householder reflector.
Deze structuur impliceert een karakteristieke veelterm van de vorm:
\begin{equation}
		det(\lambda I - Q_k) = (\lambda-1)^k \cdot det(\lambda I-F)
\end{equation}
Voor $k \geq 1$ hebben we dus steeds eigenwaarde 1 in het spectrum van $Q_k$ met algebraïsche multipliciteit $k$. 
Aangezien dit rechtstreeks voortkomt uit het diagonaal zijn van $\begin{bmatrix} I & 0\\ \end{bmatrix}^T$, is $k$ eveneens de geometrische multipliciteit. 
Een diagonale matrix is immers nooit defectief.
We kunnen dan $k$ lineair onafhankelijke eigenvectoren vinden voor $\lambda = 1$.
Dit is gemakkelijk te verifiëren.
Neem gewoon:
\begin{equation}
	v_1 = \begin{bmatrix}
		   1  \\
  		   0  \\
  		   0  \\
  		   \vdots  \\
  		   0  \\
		  \end{bmatrix}, \quad
  	v_2 = \begin{bmatrix}
		   0  \\
  		   1  \\
  		   0  \\
  		   \vdots  \\
  		   0  \\
		  \end{bmatrix}, \quad
	v_3 = \begin{bmatrix}
		   0  \\
  		   0  \\
  		   1  \\
  		   \vdots  \\
  		   0  \\
		  \end{bmatrix}, \quad  \dots \quad
	v_k = \dots
\end{equation}
Dan $Q_k v_j = v_j \quad(\forall j \leq k)$.
Deze vectoren spannen de ruimte $\mathbb{R}^k$ op waartoe alle vectoren
van de eerste $k$ rijen van een matrix $A \in \mathbb{R}^{m \times n}$ behoren (aangevuld tot dimensie $m$ met nulle. 
Vermits elke kolomvector van A tot en met rij k (en vervolgens aangevuld met nullen tot dimensie m) een lineaire combinatie is van bovenstaande $v_j$ zal een vermenigvuldiging met $Q_k$ neerkomen op een vermenigvuldiging met $\lambda = 1$. Householder laat dus de eerste $k$ rijen van de matrix $A$ ongemoeid zoals men hoopt te verwachten van een Householder transformatiematrix.\\
\\
De Householder reflector $F$ zal dan zorgen voor de nodige nullen zonder voorheen aangebrachte nullen te vernietigen.

\pagebreak
\subsection*{Opgave 3}

\subsubsection{deel a}

Voor de uitwerking van de drie gevraagde functies verwijzen we naar de bijgevoegde MATLAB bestanden.

\subsubsection{deel b}

\lipsum[9-9]

% == Hier volgt de uitwerking an adeel 2  == %
\section{Iteratieve Methoden}

\subsection*{Opgave 4}

\lipsum[3-3]

\subsection*{Opgave 5}

\subsubsection{}

Het kan aangetoond worden dat het QR-algoritme (zonder shifts) equivalent is met simultane iteratie. Het QR-algoritme maakt(veelal) gebruik van het rayleighquotient bij zijn shifts. Rayleighiteratie gebruikt het rayleighquotient afgewisseld met inverse iteratie.\\
\\
Zowel het QR-algoritme als simultane iteratie berekenen meerdere eigenwaarden.
Rayleighquotientiteratie berekent slechts de één enkele eigenwaarde, afhankelijk van de starvector.
\subsection*{Opgave 6}

\lipsum[7-7]

\subsection*{Opgave 7}

\lipsum[7-7]

\subsection*{Opgave 8}

\lipsum[7-7]

\subsection*{Opgave 9}

\lipsum[7-7]

\subsection*{Opgave 10}

\lipsum[7-7]

\subsection*{Opgave 11}

\lipsum[7-7]

\clearpage

\end{document}
