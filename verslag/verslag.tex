%%%%%%%%%%%%%%%%%%%%%%%%%%%%%%%%%%%%%%%%%%%%%%%%%%%%%%%%%%%%%%%%%%%%%%%%%%%
%%%                                                                     %%%
%%%   LaTeX template voor het verslag van P&O: Computerwetenschappen.   %%%
%%%                                                                     %%%
%%%   Opties:                                                           %%%
%%%     tt      Tussentijdsverslag                                      %%%
%%%     eind    Eindverslag                                             %%%
%%%                                                                     %%%
%%%   4 november 2013                                                   %%%
%%%   Versie 1.1                                                        %%%
%%%                                                                     %%%
%%%%%%%%%%%%%%%%%%%%%%%%%%%%%%%%%%%%%%%%%%%%%%%%%%%%%%%%%%%%%%%%%%%%%%%%%%%

\documentclass[een]{practicumverslag}
\def\thesubsubsection{\alph{subsubsection}}

%%% PACKAGES %%%
\usepackage{lipsum}
\usepackage{subcaption}


\begin{document}

% == TITELPAGINA == %
\members{Stijn Kuypers\\
         Dries De Backker} % teamleden
         
\maketitlepage


\pagebreak

% == Hier volgt de uitwerking van deel 1 == %
\section{Projectoren en de QR factorisatie}

\subsection*{Opgave 1}

\lipsum[3-3]

\subsection*{Opgave 2}

\lipsum[7-7]

\subsection*{Opgave 3}

\subsubsection{deel a}

\lipsum[8-8]

\subsubsection{deel b}

\lipsum[9-9]

% == Hier volgt de uitwerking an adeel 2  == %
\section{Iteratieve Methoden}

\subsection*{Opgave 4}

\lipsum[3-3]

\subsection*{Opgave 5}

\lipsum[7-7]

\subsection*{Opgave 6}

\lipsum[7-7]

\subsection*{Opgave 7}

\lipsum[7-7]

\subsection*{Opgave 8}

\lipsum[7-7]

\subsection*{Opgave 9}

\lipsum[7-7]

\subsection*{Opgave 10}

\lipsum[7-7]

\clearpage

\end{document}
