\documentclass[a4paper]{article}
\usepackage[dutch]{babel}
\usepackage{amsmath}
\usepackage{graphicx}

\title{Practicum NMB : Eigenwaardenproblemen}
\author{Mijn Naam}
\date{vrijdag 25 april 2014}

\newcommand{\opgave}[1]{\section*{Opgave #1}}
\newcommand{\dx}{\Delta x}
\newcommand{\dy}{\Delta y}
\newcommand{\dz}{\Delta z}
\newcommand{\dt}{\Delta t}

\begin{document}
\maketitle

\opgave{1}

De oplossing van opgave 1. Met de formule
\begin{equation*}
E = mc^2.
\end{equation*}
Of een formule in de tekst \(e^{i\pi} + 1 = 0\).
\begin{equation}
\label{vgl}
\cos^2(\alpha) + \sin^2\alpha = 1
\end{equation}

\opgave{2}

De oplossing van opgave 2. Zie figuur~\ref{figuurtje}.

\begin{figure}
\begin{center}
\includegraphics[width=0.4\textwidth]{figuurtje.eps}
\end{center}
\caption{Een figuurtje.}
\label{figuurtje}
\end{figure}


\opgave{3}

Zie tabel~\ref{tab1} en vergelijking~\eqref{vgl}.

\begin{table}
\begin{center}
\begin{tabular}{r|llc}
getal & cijfer & c & c \\\hline
een & 1 & 1 & 1 \\
tien & 10 & 10 & 10 \\
honderd & 100 & 100 & 100
\end{tabular}
\end{center}
\caption{Een tabel}
\label{tab1}
\end{table}

\end{document}
